% Created 2018-06-28 Thu 15:28
% Intended LaTeX compiler: pdflatex
\documentclass[11pt, sans]{moderncv}
\usepackage[utf8]{inputenc}
\usepackage[T1]{fontenc}
\usepackage{graphicx}
\usepackage{grffile}
\usepackage{longtable}
\usepackage{wrapfig}
\usepackage{rotating}
\usepackage[normalem]{ulem}
\usepackage{amsmath}
\usepackage{textcomp}
\usepackage{amssymb}
\usepackage{capt-of}
\usepackage{hyperref}
\moderncvstyle{classic}
\moderncvcolor{blue}
\usepackage[scale=0.75]{geometry}
\name{Tedward}{Erker}
\address{Madison, WI}
\phone[mobile]{(314)~324~6079}
\email{tedward.erker@gmail.com}                               % optional, remove / comment the line if not wanted
\homepage{stat.wisc.edu/~erker/}                         % optional, remove / comment the line if not wanted
\social[github]{tedwarderker}                              % optional, remove / comment the line if not wanted
\title{title}                               % optional, remove / comment the line if not wanted
\date{\today}
\title{~}
\hypersetup{
 pdfauthor={Tedward Erker},
 pdftitle={~},
 pdfkeywords={},
 pdfsubject={},
 pdfcreator={Emacs 25.1.1 (Org mode 9.1.7)},
 pdflang={English}}
\begin{document}

\maketitle
\section*{Education}
\label{sec:org568cc3a}
\cventry{2013--Present}{Ph.D.}{Universifty of Wisconsin--Madison}{}{\textit{3.929}}{Forestry, Department of Forest and Wildlife Ecology\newline{}Committee: Phil Townsend, Jun Zhu, Chris Kucharik, Eric Kruger, Annemarie Schneider.}
\cventry{2013--Present}{M.S.}{University of Wisconsin--Madison}{}{}{\href{https://www.stat.wisc.edu/masters-biometry}{Biometry}, Department of Statistics}
\cventry{2006--2010}{B.A.}{Washington University in St. Louis}{}{\textit{3.83}}{Environmental Studies--Ecology/Biology, Summa Cum Laude}
\section*{Experience}
\label{sec:org65e53e3}
\cventry{2015--Present}{Research Assistant}{UW-Madison}{}{}{%
\begin{itemize}
\item Map Urban Forests of Wisconsin
\begin{itemize}
\item Tested 3 machine learning algorithms to classify terabytes of imagery
\item Processed imagery in parallel at UW's Center for High Throughput Computing
\item Geospatial analysis in R and image segmentation in python.
\end{itemize}
\item Carbon Budget of Urban Forest
\begin{itemize}
\item Assessed impact of tree canopy on residential building energy use
and carbon emissions of \textasciitilde{}30,000 Madison homes.
\end{itemize}
\item Canopy Foliar Trait Mapping with \href{https://aviris-ng.jpl.nasa.gov/}{Imaging Spectroscopy}.
\begin{itemize}
\item Applied partial least squares regression models to predict foliar
canopy traits (e.g.  nitrogen content) from \href{https://aviris-ng.jpl.nasa.gov/}{imaging spectroscopy}
data
\item Explored anthropogenic and environmental drivers of trait variation
across Madison, WI.
\end{itemize}
\end{itemize}
}

\cventry{2013--2015}{Teaching Assistant}{UW-Madison}{}{}{%
\begin{itemize}
\item Statistical Methods for Bioscience II, Spring 2015
\begin{itemize}
\item Led 2 weekly discussion groups, graded homework and exams for
this graduate-level course largely covering multiple linear and
logistic regression
\item Prof. Murray Clayton.
\end{itemize}
\item Forest Ecology, Fall 2013 and Fall 2014
\begin{itemize}
\item Redesigned, created and independently implemented lab lessons in
field and computer lab for \textasciitilde{}70 students.
\item Prof. Tom Gower (2013) and Prof. Phil Townsend (2014).
\end{itemize}
\item Living With Wildlife, Spring 2014
\begin{itemize}
\item Graded journals and exams, assisted students during office hours.
\item Prof. Stan Temple.
\end{itemize}
\end{itemize}

}

\cventry{2010--2012}{Chemistry and Biology Teacher}{Confluence Prep Academy}{St. Louis}{}{
\begin{itemize}
\item Educated over 120 students in six classes daily.
\item As first year teacher, developed chemistry curriculum for new charter school integrating College Readiness Standards with Missouri Science Standards.
\item Cross-country coach
\end{itemize}
}

\cventry{2010--2012}{Corps Member}{Teach For America}{Chicago \& St. Louis}{}{
\begin{itemize}
\item Selected from over 46,000 applicants nationwide
\item Commited two years to teach in under-resourced public schools
\end{itemize}
}

\cventry{Spring 2008}{Undergraduate Teaching Assistant}{Washington University in St. Louis}{}{}{
\begin{itemize}
\item Brave New Crops, Environmental Studies 3322
\item Prof. Glenn Davis Stone
\end{itemize}
}



\section*{Awards, Grants, and Fellowships}
\label{sec:orgb10e1e4}
\cvitemwithcomment{
Jan 2018
}{
\href{http://mc-stan.org/events/}{Stan Conference Scholarship}
}{}

\cvitemwithcomment{
2015-2018
}{
\href{https://nspires.nasaprs.com/external/viewrepositorydocument/cmdocumentid=459947/solicitationId=\%7BB6CDCEA6-8EDD-A48A-FAF8-E588F66661C3\%7D/viewSolicitationDocument=1/NESSF15\%20selections.pdf}{NASA Earth and Space Science Fellowship}
}{
\$105,000
}

\cvitemwithcomment{
Sep 2016
}{
Mapping Wisconsin's Urban Tree Canopy (co-author), Wisconsin DNR
}{
\$50,000
}

\cvitemwithcomment{
Oct 2016
}{
\href{https://kb.wisc.edu/russell/page.php?id=65402}{George Kress Award for Outstanding Contribution of a Graduate Student}
}{
\$1,000
}

\cvitemwithcomment{
May 2016
}{
Travel Award, UW-Madison Department of Forest and Wildlife Ecology
}{}

\cvitemwithcomment{
Mar 2016
}{
\href{http://news.wisc.edu/cool-science-images-2016/\#\&gid=1\&pid=10}{Cool Science Image contest winner, "Madison Lakes"}
}{}

\cvitemwithcomment{
May  2010
}{
\href{http://enst.wustl.edu/program/awards}{Outstanding Overall Achievement in Environmental Studies}
}{}

\cvitemwithcomment{
Jun 2008
}{
\href{https://tyson.wustl.edu/2008}{Tyson Research Center Summer Undergraduate Research Fellowship}
}{
\$3,750
}


\section*{Presentations}
\label{sec:org6fc9fd9}
\cvitem{
Nov 2016
}{
\textbf{\href{http://pages.stat.wisc.edu/\~erker/Presentations/SAF\_20161105/saf\_presentation.html}{Mapping Urban Tree Canopy of Wisconsin}}
}
\cvitemwithcomment{}{
Society of American Foresters National Convention
}{
Madison, WI
}

\section*{Posters}
\label{sec:orgb70b176}
\cvitem{
Apr 2018
}{
\textbf{\href{http://pages.stat.wisc.edu/\~erker/Posters/NASA\_poster\_2018.jpg}{Functional and Species Diversity of Trees in Urban Streets}}
}\cvitemwithcomment{}{
NASA Biodiversity and Ecological Forecasting Team Meeting
}{
Washington, D.C.
}

\cvitem{
May 2016
}{
\textbf{\href{http://pages.stat.wisc.edu/\~erker/Posters/NASA\_poster\_2016.jp2}{How Does the Urban Forest Affect the Urban Heat Island and Building Energy Use?}}
}\cvitemwithcomment{}{
NASA Biodiversity and Ecological Forecasting Team Meeting
}{
Silver Springs, MD.
}

\section*{Teaching Experience}
\label{sec:orga867a1b}
\subsection*{Spring 2015}
\label{sec:orgd060dfb}
\textbf{Statistical Methods for Bioscience II Teaching Assistant}

Led 2 weekly discussion groups, graded homework and exams for this
graduate-level course. Prof. Murray Clayton.
;
\subsection*{Fall 2013 and Fall 2014}
\label{sec:org879fe44}
\textbf{Forest Ecology Teaching Assistant}

Redesigned, created and independently implemented lab lessons in field and computer lab for \textasciitilde{}70
students.  Prof. Tom Gower (2013) and Prof. Phil Townsend (2014).

\subsection*{Spring 2014}
\label{sec:org8012acc}
\textbf{Living With Wildlife Teaching Assistant}

Graded journals and exams, assisted students during office hours.
Prof. Stan Temple.

\subsection*{August 2010 - May 2012}
\label{sec:org16fbfae}
\textbf{High School Chemistry and Biology Teacher}

Educated over 120 students in six classes daily. As first year
teacher, developed chemistry curriculum for new charter school
integrating College Readiness Standards with Missouri Science
Standards. Cross-country coach. \href{https://www.google.com/search?q=Confluence+Preparatory+Academy+St.+Louis\&oq=Confluence+Prep+Academy+St.+Louis\&aqs=chrome..69i57.7294j0j8\&sourceid=chrome\&ie=UTF-8\#q=Confluence+Prep+Academy+High+School+St.+Louis}{Confluence Preparatory Academy}. St. Louis, MO.

\subsection*{June 2010 - May 2012}
\label{sec:org19d9832}
\textbf{\href{https://www.teachforamerica.org/}{Teach For America} Corps Member}

Selected from over 46,000 applicants nationwide to join the national
teacher corps of recent college graduates who commit two years to
teach in under-resourced public schools.  Chicago, IL \& St. Louis, MO.

\section*{Mentoring}
\label{sec:org87e4b34}

My undergraduate mentees research a topic, collect new data, perform statistical
analyses, and write final papers to complete small research projects.  They
create posters or presentations to share their work.

\subsection*{Fall 2017}
\label{sec:org658fee9}
\textbf{Cheyenne Brandt}

Undergraduate Research Project: Effect of Leaf Area and Tree Canopy on the Urban
Heat Island of Madison, WI.

\subsection*{Fall 2015}
\label{sec:org7f076fe}
\textbf{Bobby Shepherd}

Undergraduate Research Project: Investigating the influence of the
urban heat island on autumn phenology of Norway Maple (\emph{Acer
platanoides}) with smartphone hemispherical photos.
\section*{Relevant Work Experience}
\label{sec:org17df3e7}

\subsection*{August 2013 - December 2014}
\label{sec:org30f3f0b}
\textbf{Arborist}

Worked part time as ground crew, hauling brush and aiding climber.
Developed online Wisconsin tree species identification application.
\href{http://www.urbantreealliance.org/}{Urban Tree Alliance}, Non-profit. Madison, WI.

\subsection*{February 2013 - August 2013}
\label{sec:org6522ab2}
\textbf{Arborist}

Performed ground crew work, climbed, and as certified pesticide
applicator treated for a number of pests including the emerald ash
borer. \href{http://atetreecare.com/}{American Tree Experts}.  New Berlin, WI

\subsection*{March 2007 - May 2010}
\label{sec:orgf669385}
\textbf{Greenhouse Assistant}

Water, transplant, propagate plants; maintain greenhouse. Work-Study
at \href{http://biology4.wustl.edu/greenhouse/index.html}{Washington University Plant Research Facility}. St. Louis, MO.

\subsection*{April - August 2009}
\label{sec:orge6b9b46}
\textbf{Farm Education Intern and Farmer}

Organized and guided trips of 16-40 students at farm and wilderness
education center. Managed 3⁄4 acre garden and cared for sheep, goats,
chickens, pigs, and cows as part of farm team.  \href{https://farmandwilderness.org/}{Farm And Wilderness
Summer Camps}. Plymouth, VT.

\subsection*{May - August 2008}
\label{sec:orgde176a6}
\textbf{Research Assistant}

Sampled vegetation, identified over 100 plant species as part of study
to explore phylogenic relationships in invasiveness.  \href{https://tyson.wustl.edu/2008}{Tyson Research
Center}. Eureka, MO.
\section*{Professional Affiliations}
\label{sec:org2a24770}
Society of American Foresters, 2016--Present

\section*{Languages}
\label{sec:org3662c49}
Spoken: English, Spanish

Computing: R

\section*{Service to the Department and University}
\label{sec:org64fc077}
\subsection*{Oct 2015 - Present}
\label{sec:org711251b}
\textbf{Graduate Student Representative, Department of Forest and Wildlife
Ecology}
\subsection*{Jul 2017}
\label{sec:orgcf4dcb7}
\textbf{Software Carpentry Volunteer}
\section*{Service to Community}
\label{sec:orgb64635c}
\subsection*{Fall 2014 and Fall 2015}
\label{sec:org62e49f9}
\textbf{Guest Lab Instructor, Sustainability by the Numbers, Shabazz High School}

\subsection*{Fall 2017}
\label{sec:org15c4ded}
\textbf{Guest Lab Instructor, AP Environmental Studies, East High School}


\section*{Graduate Coursework}
\label{sec:org6dfa009}
\begin{center}
\begin{tabular}{lll}
Semester & Course & Grade\\
\hline
F 2013 & Diseases of Trees and Shrubs & A\\
 & Tree Physiology & A\\
 & Statistical Methods for Bioscience I & A\\
S 2014 & Inquiry-Based Biology Teaching & A\\
 & Intermediate Data Analysis with R & A\\
 & Principles of Silviculture & S\\
 & Statistical Methods for Bioscience II & A\\
 & Teaching Biology: Special Topics & A\\
 & Advanced Data Analysis with R & A\\
Su 2014 & Calculus--Functions of Variables & S\\
F 2014 & Field Methods in Remote Sensing & A\\
 & Environmental Biophysics & A\\
 & Intro Mathematical Statistics I & A\\
S 2015 & Tools for Reproducible Research & A\\
 & Remote Sensing Digital Image Processing & A\\
 & Intro Mathematical Statistics II & AB\\
 & Teaching Statistics & A\\
Su 2015 & Statistical Consulting & A\\
F 2015 & Statistical Methods-Spatial Data & AB\\
S 2016 & Multilevel Models & A\\
S 2017 & Ecosystem Concepts & B\\
\end{tabular}
\end{center}
\end{document}