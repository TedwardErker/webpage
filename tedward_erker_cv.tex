% Created 2019-10-25 Fri 20:44
% Intended LaTeX compiler: pdflatex
\documentclass[11pt, sans]{moderncv}
\usepackage[utf8]{inputenc}
\usepackage[T1]{fontenc}
\usepackage{graphicx}
\usepackage{grffile}
\usepackage{longtable}
\usepackage{wrapfig}
\usepackage{rotating}
\usepackage[normalem]{ulem}
\usepackage{amsmath}
\usepackage{textcomp}
\usepackage{amssymb}
\usepackage{capt-of}
\usepackage{hyperref}
\moderncvstyle{classic}
\moderncvcolor{blue}
\usepackage[scale=0.75]{geometry}
\name{Tedward}{Erker}
\address{Madison, WI}
\phone[mobile]{(314)~324~6079}
\email{tedward.erker@gmail.com}                               % optional, remove / comment the line if not wanted
\homepage{stat.wisc.edu/~erker/}                         % optional, remove / comment the line if not wanted
\social[github]{tedwarderker}                              % optional, remove / comment the line if not wanted
\title{title}                               % optional, remove / comment the line if not wanted
\date{\today}
\title{~}
\hypersetup{
 pdfauthor={Tedward Erker},
 pdftitle={~},
 pdfkeywords={},
 pdfsubject={},
 pdfcreator={Emacs 25.2.1 (Org mode 9.1.7)}, 
 pdflang={English}}
\begin{document}

\maketitle
\section*{Education}
\label{sec:org133c6a4}
\cventry{2013--Present}{Ph.D.}{Universifty of Wisconsin--Madison}{}{\textit{3.929}}{Forestry, Department of Forest and Wildlife Ecology\newline{}Committee: Phil Townsend, Jun Zhu, Chris Kucharik, Eric Kruger, Annemarie Schneider.}
\cventry{2013--2018}{M.S.}{University of Wisconsin--Madison}{}{}{\href{https://www.stat.wisc.edu/masters-biometry}{Biometry}, Department of Statistics}
\cventry{2006--2010}{B.A.}{Washington University in St. Louis}{}{\textit{3.83}}{Environmental Studies--Ecology/Biology, Summa Cum Laude}
\section*{Experience}
\label{sec:orgc608778}
\cventry{2015--Present}{Research Assistant}{UW-Madison}{}{}{%
\begin{itemize}
\item Urban Forest Cover of Wisconsin
\begin{itemize}
\item Tested 3 machine learning algorithms to classify terabytes of 1 m
NAIP imagery
\item Processed imagery in parallel at UW's Center for High Throughput Computing
\item Geospatial analysis in R and image segmentation in python.
\item Product available for analysis through \href{https://landscape.itreetools.org/maps/}{iTree landscape}.
\end{itemize}
\item Carbon Budget of Urban Forest
\begin{itemize}
\item Assessed impact of tree canopy on residential building energy use
and carbon emissions of \textasciitilde{}30,000 Madison homes.
\end{itemize}
\item Growth Equations for Urban Trees
\begin{itemize}
\item Developed a bayesian nonlinear multilevel model of urban tree growth
\item Method makes more biologically realistic predictions than
\href{https://www.fs.fed.us/psw/publications/documents/psw\_gtr253/psw\_gtr\_253.pdf}{exisiting urban tree growth equations}
\item Method allows for predictions for trees of unobserved species or
cities while accounting for uncertainty.
\end{itemize}
\item Canopy Foliar Trait Mapping with \href{https://aviris-ng.jpl.nasa.gov/}{Imaging Spectroscopy}.
\begin{itemize}
\item Applied partial least squares regression models to predict foliar
canopy traits (e.g.  nitrogen content) from \href{https://aviris-ng.jpl.nasa.gov/}{imaging spectroscopy}
data
\item Explored anthropogenic and environmental drivers of trait variation
across Madison, WI.
\end{itemize}
\end{itemize}
}

\cventry{2013--2015, 2018}{Teaching Assistant}{UW-Madison}{}{}{
\begin{itemize}
\item Forest Ecology, Fall 2013, Fall 2014, Fall 2018
\begin{itemize}
\item Redesigned, created and independently implemented lab lessons in
field and computer lab for \textasciitilde{}70 students (2014).
\item Developed cases, led student discussions, and graded weekly papers (2018).
\item Prof. Tom Gower (2013) and Prof. Phil Townsend (2014, 2018).
\end{itemize}
\item Statistical Methods for Bioscience II, Spring 2015
\begin{itemize}
\item Led 2 weekly discussion groups, graded homework and exams for
this graduate-level course largely covering multiple linear and
logistic regression
\item Prof. Murray Clayton.
\end{itemize}
\item Living With Wildlife, Spring 2014
\begin{itemize}
\item Graded journals and exams, assisted students during office hours.
\item Prof. Stan Temple.
\end{itemize}
\end{itemize}
}

\cventry{2013--2014}{Arborist}{\href{http://www.urbantreealliance.org/}{Urban Tree Alliance}}{Madison, WI}{}{
\begin{itemize}
\item Worked part time as ground crew, hauling brush and aiding climber.
\item Developed online Wisconsin tree species identification application.
\end{itemize}
}

\cventry{Feb--Jul 2013}{Arborist}{\href{http://atetreecare.com/}{American Tree Experts}}{New Berlin, WI}{}{
\begin{itemize}
\item Performed ground crew work and climbed for pruning and removals
\item As certified pesticide applicator, treated for a number of pests
including the emerald ash borer.
\end{itemize}
}

\cventry{2010--2012}{Chemistry and Biology Teacher}{Confluence Prep Academy}{St. Louis}{}{
\begin{itemize}
\item Educated over 120 students in six classes daily.
\item As first year teacher, developed chemistry curriculum for new charter school integrating College Readiness Standards with Missouri Science Standards.
\item Cross-country coach
\end{itemize}
}

\cventry{2010--2012}{Corps Member}{Teach For America}{Chicago \& St. Louis}{}{
\begin{itemize}
\item Selected from over 46,000 applicants nationwide
\item Commited two years to teach in under-resourced public schools
\end{itemize}
}

\cventry{2007--2010}{Greenhouse Assistant}{\href{http://biology4.wustl.edu/greenhouse/index.html}{Wash. U. Plant Research Facility}}{St. Louis, MO}{}{
\begin{itemize}
\item Water, transplant, and propagate plants; maintain greenhouse.
\item Work-Study
\end{itemize}
}

\cventry{Apr--Aug 2009}{Farm Education Intern and Farmer}{\href{https://farmandwilderness.org/}{Farm And Wilderness}}{Plymouth, VT}{}{
\begin{itemize}
\item Organized and guided trips of 16-40 students at farm and wilderness education center.
\item Managed 3⁄4 acre garden and cared for sheep, goats, chickens, pigs, and cows as part of farm team.
\end{itemize}
}

\cventry{May--Aug 2008}{Research Assistant}{\href{}{Tyson Research Center}}{Eureka, MO}{}{
\begin{itemize}
\item Sampled vegetation, identified over 100 plant species as part of
\href{https://esajournals.onlinelibrary.wiley.com/doi/abs/10.1890/12-1310.1}{study} to explore phylogenic relationships in invasiveness.
\end{itemize}
}

\cventry{Jan--May 2008}{Undergraduate Teaching Assistant}{Washington University in St. Louis}{}{}{
\begin{itemize}
\item Brave New Crops, Environmental Studies 3322
\item Prof. Glenn Davis Stone
\end{itemize}
}

\section*{Awards, Grants, and Fellowships}
\label{sec:org18cdc15}
\cvitemwithcomment{
Jan 2018
}{
\href{http://mc-stan.org/events/}{Stan Conference Scholarship}
}{}

\cvitemwithcomment{
2015-2018
}{
\href{https://nspires.nasaprs.com/external/viewrepositorydocument/cmdocumentid=459947/solicitationId=\%7BB6CDCEA6-8EDD-A48A-FAF8-E588F66661C3\%7D/viewSolicitationDocument=1/NESSF15\%20selections.pdf}{NASA Earth and Space Science Fellowship}
}{
\$105,000
}

\cvitemwithcomment{
Sep 2016
}{
Mapping Wisconsin's Urban Tree Canopy (co-author), Wisconsin DNR
}{
\$50,000
}

\cvitemwithcomment{
Oct 2016
}{
\href{https://kb.wisc.edu/russell/page.php?id=65402}{George Kress Award for Outstanding Contribution of a Graduate Student}
}{
\$1,000
}

\cvitemwithcomment{
May 2016
}{
Travel Award, UW-Madison Department of Forest and Wildlife Ecology
}{}

\cvitemwithcomment{
Mar 2016
}{
\href{http://news.wisc.edu/cool-science-images-2016/\#\&gid=1\&pid=10}{Cool Science Image contest winner, "Madison Lakes"}
}{}

\cvitemwithcomment{
May  2010
}{
\href{http://enst.wustl.edu/program/awards}{Outstanding Overall Achievement in Environmental Studies}
}{}

\cvitemwithcomment{
Jun 2008
}{
\href{https://tyson.wustl.edu/2008}{Tyson Research Center Summer Undergraduate Research Fellowship}
}{
\$3,750
}

\section*{Publications}
\label{sec:org274e710}
\cvitem{}{Erker T., Townsend P.A., Wang L., Lorentz L., and Stoltman A. \textit{(in review)}. A statewide urban tree canopy mapping method. \textit{Remote Sensing of Environment}}
\cvitem{}{Erker T., Townsend P.A. \textit{(in review)}. Trees in many US cities may indirectly increase atmospheric carbon. \textit{Environmental Research Letters}}
\cvitem{}{Erker T., Townsend P.A., Zhu J., \textit{(in prep).} A bayesian nonlinear multilevel model of urban tree growth }
\cvitem{}{Erker T., Townsend P.A., \textit{(in prep).} Environmental drivers of urban tree canopy foliar traits derived from imaging spectroscopy}

\section*{Presentations}
\label{sec:org8c09ca6}
\cvitem{
Dec 2018
}{
\textbf{\href{http://pages.stat.wisc.edu/\~erker/Presentations/Biometry\_Defense\_20181203/allo\_presentation.html}{A Bayesian Nonlinear Multilevel Model of Urban Tree Growth}}
}
\cvitemwithcomment{}{
Biometry MS Defense
}{
Madison, WI
}

\cvitem{
Nov 2016
}{
\textbf{\href{http://pages.stat.wisc.edu/\~erker/Presentations/SAF\_20161105/saf\_presentation.html}{Mapping Urban Tree Canopy of Wisconsin}}
}
\cvitemwithcomment{}{
Society of American Foresters National Convention
}{
Madison, WI
}

\section*{Posters}
\label{sec:orgb4f7442}
\cvitem{
Dec 2018
}{
\href{http://pages.stat.wisc.edu/\~erker/Posters/erker\_energy\_agu\_2018.jpg}{\textbf{Urban shade trees may be an atmospheric carbon source for much of the
US}}
}\cvitemwithcomment{}{
American Geophysical Union Fall Meeting 2018
}{
Washington, D.C.
}

\cvitem{
Apr 2018
}{
\textbf{\href{http://pages.stat.wisc.edu/\~erker/Posters/NASA\_poster\_2018.jpg}{Functional and Species Diversity of Trees in Urban Streets}}
}\cvitemwithcomment{}{
NASA Biodiversity and Ecological Forecasting Team Meeting
}{
Washington, D.C.
}

\cvitem{
May 2016
}{
\textbf{\href{http://pages.stat.wisc.edu/\~erker/Posters/NASA\_poster\_2016.jp2}{How Does the Urban Forest Affect the Urban Heat Island and Building Energy Use?}}
}\cvitemwithcomment{}{
NASA Biodiversity and Ecological Forecasting Team Meeting
}{
Silver Springs, MD.
}

\section*{Mentoring}
\label{sec:org5e97f20}
\newline{}\cvitemwithcomment{
2017
}{
Cheyenne Brandt
}{
Effect of Leaf Area and Tree Canopy on the Urban Heat Island of Madison, WI.
}

\cvitemwithcomment{
2015
}{
Bobby Shepherd
}{
Investigating the influence of the urban heat island on autumn
phenology of \emph{Acer platanoides} with smartphone hemispherical photos.
}

\section*{Professional Service}
\label{sec:orge8fe463}
Referee/Reviewer: Remote Sensing of Environment
\section*{Professional Affiliations}
\label{sec:org2eebe1b}
\cvitem{2016--Present}{Society of American Foresters}
\cvitem{2018--Present}{American Geophysical Union}
\section*{Languages}
\label{sec:orgde18f8a}
\cvitem{Spoken:}{English, Spanish}
\cvitem{Programming:}{R, Python, Stan}
\section*{Service to the Department and University}
\label{sec:org0d970c1}
\cvitemwithcomment{2015--2018}{Graduate Student Representative}{Department of Forest and Wildlife Ecology}
\cvitemwithcomment{2017}{\href{https://software-carpentry.org/}{Software Carpentry Volunteer}}{UW-Madison}
\section*{Service to Community}
\label{sec:org8b66f1d}
\cvitemwithcomment{2014, 2015}{Guest Lab Instructor, Sustainability by the Numbers}{Shabazz High School}
\cvitemwithcomment{2017}{Guest Lab Instructor, AP Environmental Studies}{East High School}
\section*{Graduate Coursework}
\label{sec:orgf40220f}
\begin{center}
\begin{tabular}{lll}
Semester ~~ & Course                             ~ & Grade\\
\hline
F 2013 & Diseases of Trees and Shrubs & A\\
 & Tree Physiology & A\\
 & Statistical Methods for Bioscience I & A\\
S 2014 & Inquiry-Based Biology Teaching & A\\
 & Intermediate Data Analysis with R & A\\
 & Principles of Silviculture & S\\
 & Statistical Methods for Bioscience II & A\\
 & Teaching Biology: Special Topics & A\\
 & Advanced Data Analysis with R & A\\
Su 2014 & Calculus--Functions of Variables & S\\
F 2014 & Field Methods in Remote Sensing & A\\
 & Environmental Biophysics & A\\
 & Intro Mathematical Statistics I & A\\
S 2015 & Tools for Reproducible Research & A\\
 & Remote Sensing Digital Image Processing~~~~ & A\\
 & Intro Mathematical Statistics II & AB\\
 & Teaching Statistics & A\\
Su 2015 & Statistical Consulting & A\\
F 2015 & Statistical Methods-Spatial Data & AB\\
S 2016 & Multilevel Models & A\\
S 2017 & Ecosystem Concepts & B\\
\end{tabular}
\end{center}

\section*{Workshops}
\label{sec:orge44d354}
\cvitemwithcomment{2018}{SESYNC: Urban Woodlands Pursuit}{}
\cvitemwithcomment{2017}{Hierarchical Modeling and Analysis of Spatial-Temporal Data}{Andrew Finley}
\cvitemwithcomment{2016}{Software Carpentry}{}
\end{document}
