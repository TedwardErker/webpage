% Created 2018-06-25 Mon 13:52
% Intended LaTeX compiler: pdflatex
\documentclass[11pt, sans]{moderncv}
\usepackage[utf8]{inputenc}
\usepackage[T1]{fontenc}
\usepackage{graphicx}
\usepackage{grffile}
\usepackage{longtable}
\usepackage{wrapfig}
\usepackage{rotating}
\usepackage[normalem]{ulem}
\usepackage{amsmath}
\usepackage{textcomp}
\usepackage{amssymb}
\usepackage{capt-of}
\usepackage{hyperref}
\moderncvstyle{classic}
\moderncvcolor{blue}
\usepackage[scale=0.75]{geometry}
\name{Tedward}{Erker}
\address{Madison, WI}
\phone[mobile]{(314)~324~6079}
\email{tedward.erker@gmail.com}                               % optional, remove / comment the line if not wanted
\homepage{stat.wisc.edu/~erker/}                         % optional, remove / comment the line if not wanted
\social[github]{tedwarderker}                              % optional, remove / comment the line if not wanted
\title{title}                               % optional, remove / comment the line if not wanted
\date{\today}
\title{~}
\hypersetup{
 pdfauthor={Tedward Erker},
 pdftitle={~},
 pdfkeywords={},
 pdfsubject={},
 pdfcreator={Emacs 25.1.1 (Org mode 9.1.7)},
 pdflang={English}}
\begin{document}

\maketitle
\section*{Summary of Qualifications and Skills}
\label{sec:org853c940}
\href{https://www.stat.wisc.edu/masters-biometry}{Biometry M.S.} and Forestry Ph.D. (expected spring 2019) with 5 years
of research and data analysis experience and 2 years of high school
teaching experience.  Comfortable with a wide range of statistical
methods including generalized linear models and tree-based
methods. Driven to understand complex problems and distill key
findings for nonexpert audiences via fully reproducible reports and
compelling figures.  Independently motivated. A positive, constructive
team member and leader.  Passionate about working in education.
\section*{Experience}
\label{sec:orgd406d64}
\cventry{2015--Present}{Research Assistant}{UW-Madison}{}{}{%
\begin{itemize}
\item Map Urban Forests of Wisconsin
\begin{itemize}
\item Tested 3 machine learning algorithms to classify nearly a terabyte of imagery
\item Processed imagery in parallel at UW's Center for High Throughput Computing
\item Geospatial analysis in R and image segmentation in python.
\end{itemize}
\item Carbon Budget of Urban Forest
\begin{itemize}
\item Assessed impact of tree canopy on residential building energy use
and carbon emissions of \textasciitilde{}30,000 Madison homes.
\end{itemize}
\item Canopy Foliar Trait Mapping with \href{https://aviris-ng.jpl.nasa.gov/}{Imaging Spectroscopy}.
\begin{itemize}
\item Applied partial least squares regression models to predict foliar
canopy traits (e.g.  nitrogen content) from \href{https://aviris-ng.jpl.nasa.gov/}{imaging spectroscopy}
data
\item Explored anthropogenic and environmental drivers of trait variation
across Madison, WI.
\end{itemize}
\end{itemize}
}

\cventry{2013--2015}{Teaching Assistant}{UW-Madison}{}{}{%
\begin{itemize}
\item Statistical Methods for Bioscience II, Spring 2015
\begin{itemize}
\item Led 2 weekly discussion groups, graded homework and exams for
  this graduate-level course largely covering multiple linear and
logistic regression
\end{itemize}
\item Forest Ecology, Fall 2013 and Fall 2014
\item Living With Wildlife, Spring 2014
\end{itemize}
}

\cventry{2010--2012}{Chemistry and Biology Teacher}{Confluence Prep Academy}{St. Louis}{}{
\begin{itemize}
\item Educated over 120 students in six classes daily.
\item As first year teacher, developed chemistry curriculum for new charter school integrating College Readiness Standards with Missouri Science Standards.
\end{itemize}
}

\cventry{2010--2012}{Corps Member}{Teach For America}{Chicago \& St. Louis}{}{
}

\section*{Education}
\label{sec:org52ca837}
\cventry{2013--Present}{Ph.D.}{University of Wisconsin--Madison}{}{\textit{3.929}}{Forestry, Department of Forest and Wildlife Ecology}
\cventry{2013--Present}{M.S.}{University of Wisconsin--Madison}{}{}{\href{https://www.stat.wisc.edu/masters-biometry}{Biometry}, Department of Statistics}
\cventry{2006--2008 2009-2010}{B.A.}{Washington University in St. Louis}{}{\textit{3.83}}{Environmental Studies--Ecology/Biology, Summa Cum Laude}

\section*{Relevant graduate coursework}
\label{sec:orgc08515d}
\cvlistdoubleitem{Tools for Reproducible Research}{Advanced Data Analysis with R}
\cvlistdoubleitem{Statistical Methods-Spatial Data}{Multilevel Models}
\cvlistdoubleitem{Intro Mathematical Statistics I \& II}{Statistical Meth. for Bioscience I \& II}
\cvlistdoubleitem{Teaching Statistics}{Statistical Consulting}

\section*{Skills}
\label{sec:org39c2da3}
\cvitemwithcomment{Writing}{}{1 scientific paper in review; over \$150,00 in proposals}
\cvitemwithcomment{Presenting}{}{2 scientific posters, 1 academic presentation, 4 years of teaching}
\cvitemwithcomment{Data Display}{}{Daily use of grammar of graphics in R's ggplot2}
\cvitemwithcomment{Statistical Analysis}{}{GLMs, GAMs, multilevel models, shrinkage and dimension reduction, tree-based methods, dependent data in R and some Stan}
\cvitemwithcomment{Computing}{}{R, python, webscraping, emacs org mode, unix command line, version control (git)}

\clearpage
\recipient{Education Analytics Recruitment Team}{Education Analytics, Inc.\\131 West Wilson Street, Suite 200\\Madison, WI 53703}
\date{\today}
\opening{To Research Analyst Hiring Manager,}
\closing{My best,}
\enclosure[Attached]{resume}          % use an optional argument to use a string other than "Enclosure", or redefine \enclname
\makelettertitle

Lorem ipsum dolor sit amet, consectetur adipiscing elit. Duis
ullamcorper neque sit amet lectus facilisis sed luctus nisl
iaculis. Vivamus at neque arcu, sed tempor quam. Curabitur pharetra
tincidunt tincidunt. Morbi volutpat feugiat mauris, quis tempor neque
vehicula volutpat. Duis tristique justo vel massa fermentum
accumsan. Mauris ante elit, feugiat vestibulum tempor eget, eleifend
ac ipsum. Donec scelerisque lobortis ipsum eu vestibulum. Pellentesque
vel massa at felis accumsan rhoncus.


Suspendisse commodo, massa eu congue tincidunt, elit mauris
pellentesque orci, cursus tempor odio nisl euismod augue. Aliquam
adipiscing nibh ut odio sodales et pulvinar tortor laoreet. Mauris a
accumsan ligula. Class aptent taciti sociosqu ad litora torquent per
conubia nostra, per inceptos himenaeos. Suspendisse vulputate sem
vehicula ipsum varius nec tempus dui dapibus. Phasellus et est urna,
ut auctor erat. Sed tincidunt odio id odio aliquam mattis. Donec
sapien nulla, feugiat eget adipiscing sit amet, lacinia ut
dolor. Phasellus tincidunt, leo a fringilla consectetur, felis diam
aliquam urna, vitae aliquet lectus orci nec velit. Vivamus dapibus
varius blandit.


Duis sit amet magna ante, at sodales diam. Aenean consectetur porta
risus et sagittis. Ut interdum, enim varius pellentesque tincidunt,
magna libero sodales tortor, ut fermentum nunc metus a ante. Vivamus
odio leo, tincidunt eu luctus ut, sollicitudin sit amet metus. Nunc
sed orci lectus. Ut sodales magna sed velit volutpat sit amet pulvinar
diam venenatis.


Albert Einstein discovered that \(e=mc^2\) in 1905.


\makeletterclosing
\end{document}