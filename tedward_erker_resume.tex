% Created 2018-06-25 Mon 14:15
% Intended LaTeX compiler: pdflatex
\documentclass[11pt, sans]{moderncv}
\usepackage[utf8]{inputenc}
\usepackage[T1]{fontenc}
\usepackage{graphicx}
\usepackage{grffile}
\usepackage{longtable}
\usepackage{wrapfig}
\usepackage{rotating}
\usepackage[normalem]{ulem}
\usepackage{amsmath}
\usepackage{textcomp}
\usepackage{amssymb}
\usepackage{capt-of}
\usepackage{hyperref}
\moderncvstyle{classic}
\moderncvcolor{blue}
\usepackage[scale=0.75]{geometry}
\name{Tedward}{Erker}
\address{Madison, WI}
\phone[mobile]{(314)~324~6079}
\email{tedward.erker@gmail.com}                               % optional, remove / comment the line if not wanted
\homepage{stat.wisc.edu/~erker/}                         % optional, remove / comment the line if not wanted
\social[github]{tedwarderker}                              % optional, remove / comment the line if not wanted
\title{title}                               % optional, remove / comment the line if not wanted
\date{\today}
\title{~}
\hypersetup{
 pdfauthor={Tedward Erker},
 pdftitle={~},
 pdfkeywords={},
 pdfsubject={},
 pdfcreator={Emacs 25.1.1 (Org mode 9.1.7)},
 pdflang={English}}
\begin{document}

\maketitle
\section*{Summary of Qualifications and Skills}
\label{sec:org0160b60}
\href{https://www.stat.wisc.edu/masters-biometry}{Biometry M.S.} and Forestry Ph.D. (expected spring 2019) with 5 years
of research and data analysis experience and 2 years of high school
teaching experience.  Comfortable with a wide range of statistical
methods including generalized linear models and tree-based
methods. Driven to understand complex problems and distill key
findings for nonexpert audiences via fully reproducible reports and
compelling figures.  Independently motivated. A positive, constructive
team member and leader.  Passionate about working in education.
\section*{Experience}
\label{sec:org4b0119d}
\cventry{2015--Present}{Research Assistant}{UW-Madison}{}{}{%
\begin{itemize}
\item Map Urban Forests of Wisconsin
\begin{itemize}
\item Tested 3 machine learning algorithms to classify terabytes of imagery
\item Processed imagery in parallel at UW's Center for High Throughput Computing
\item Geospatial analysis in R and image segmentation in python.
\end{itemize}
\item Carbon Budget of Urban Forest
\begin{itemize}
\item Assessed impact of tree canopy on residential building energy use
and carbon emissions of \textasciitilde{}30,000 Madison homes.
\end{itemize}
\item Canopy Foliar Trait Mapping with \href{https://aviris-ng.jpl.nasa.gov/}{Imaging Spectroscopy}.
\begin{itemize}
\item Applied partial least squares regression models to predict foliar
canopy traits (e.g.  nitrogen content) from \href{https://aviris-ng.jpl.nasa.gov/}{imaging spectroscopy}
data
\item Explored anthropogenic and environmental drivers of trait variation
across Madison, WI.
\end{itemize}
\end{itemize}
}

\cventry{2013--2015}{Teaching Assistant}{UW-Madison}{}{}{%
\begin{itemize}
\item Statistical Methods for Bioscience II, Spring 2015
\begin{itemize}
\item Led 2 weekly discussion groups, graded homework and exams for
  this graduate-level course largely covering multiple linear and
logistic regression
\end{itemize}
\item Forest Ecology, Fall 2013 and Fall 2014
\item Living With Wildlife, Spring 2014
\end{itemize}
}

\cventry{2010--2012}{Chemistry and Biology Teacher}{Confluence Prep Academy}{St. Louis}{}{
\begin{itemize}
\item Educated over 120 students in six classes daily.
\item As first year teacher, developed chemistry curriculum for new charter school integrating College Readiness Standards with Missouri Science Standards.
\end{itemize}
}

\cventry{2010--2012}{Corps Member}{Teach For America}{Chicago \& St. Louis}{}{
}

\section*{Education}
\label{sec:org2d1f938}
\cventry{2013--Present}{Ph.D.}{University of Wisconsin--Madison}{}{\textit{3.929}}{Forestry, Department of Forest and Wildlife Ecology}
\cventry{2013--Present}{M.S.}{University of Wisconsin--Madison}{}{}{\href{https://www.stat.wisc.edu/masters-biometry}{Biometry}, Department of Statistics}
\cventry{2006--2008 2009-2010}{B.A.}{Washington University in St. Louis}{}{\textit{3.83}}{Environmental Studies--Ecology/Biology, Summa Cum Laude}

\section*{Relevant graduate coursework}
\label{sec:org5b325bf}
\cvlistdoubleitem{Tools for Reproducible Research}{Advanced Data Analysis with R}
\cvlistdoubleitem{Statistical Methods-Spatial Data}{Multilevel Models}
\cvlistdoubleitem{Intro Mathematical Statistics I \& II}{Statistical Meth. for Bioscience I \& II}
\cvlistdoubleitem{Teaching Statistics}{Statistical Consulting}

\section*{Skills}
\label{sec:org5de9a47}
\cvitemwithcomment{Writing}{}{1 scientific paper in review; over \$150,00 in proposals}
\cvitemwithcomment{Presenting}{}{2 scientific posters, 1 academic presentation, 4 years of teaching}
\cvitemwithcomment{Data Display}{}{Daily use of grammar of graphics in R's ggplot2}
\cvitemwithcomment{Statistical Analysis}{}{GLMs, GAMs, multilevel models, shrinkage and dimension reduction, tree-based methods, dependent data in R and some Stan}
\cvitemwithcomment{Computing}{}{R, python, webscraping, emacs org mode, unix command line, version control (git)}
\cvitemwithcomment{Mentoring}{}{2 undergraduate research assistants, 4 years of teaching}

\clearpage
\recipient{Education Analytics Recruitment Team}{Education Analytics, Inc.\\131 West Wilson Street, Suite 200\\Madison, WI 53703}
\date{\today}
\opening{To Research Analyst Hiring Manager,}
\closing{My best,}
\enclosure[Attached]{resume}          % use an optional argument to use a string other than "Enclosure", or redefine \enclname
\makelettertitle

My first attempt at data-driven improvement in student achievement was
in 2010.  As a first year biology and chemistry teacher at a charter
school in St. Louis, I wanted my students to grow at least 3 points on
the ACT, an ambitious push towards college readiness.  I administered
tests quarterly, tracked students' performance, and created daily
practice problems for target standards.  Working with this data, I
encountered the challenges inherent in measuring student learning.
How reliable are the tests that I'm putting together from old ACTs?
What to do about missing students?  How much of my students'
performance is even atributable to me, the teacher?  And how can I use
these results to make my teaching more effective?

I loved my high school students, but I wanted to teach college level
courses.  So in 2013 I moved to Madison to pursue a PhD in Forestry
and a Masters in Biometry.  My proclivity for data analysis followed
me and was honed at the university.  The highlights of my statistical
training were a multilevel modeling course, being the teaching
assistant for a graduate level statistics course for natural science
majors, and statistical consulting.  Relevant skills from my PhD work
include working with large datasets (terabytes of imagery) in R, using
machine learning algorithms (e.g. random forests), writing literate
programs for reproducible research, and mastering the visual display
of quantitative information for effective communication.

As I am nearing the completion of my PhD and considering the needs of
my family, I am reconsidering the academic career path.  I'm exploring
the option of staying in Madison but want to be sure I can make full
use of my skills and work for the common good.  I believe, like
Education Analytics, that the intelligent use of the correct data will
help create a better education system, and by extension, a better
world.  I want to join Education Analytics and apply my skills to
build this world, one district at a time.

\makeletterclosing
\end{document}