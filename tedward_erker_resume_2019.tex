% Created 2019-05-01 Wed 09:57
% Intended LaTeX compiler: pdflatex
\documentclass[11pt, sans]{moderncv}
\usepackage[utf8]{inputenc}
\usepackage[T1]{fontenc}
\usepackage{graphicx}
\usepackage{grffile}
\usepackage{longtable}
\usepackage{wrapfig}
\usepackage{rotating}
\usepackage[normalem]{ulem}
\usepackage{amsmath}
\usepackage{textcomp}
\usepackage{amssymb}
\usepackage{capt-of}
\usepackage{hyperref}
\moderncvstyle{classic}
\moderncvcolor{blue}
\usepackage[scale=0.75]{geometry}
\name{Tedward}{Erker}
% \address{Madison, WI}
% \phone[mobile]{(314)~324~6079}
\email{tedward.erker@gmail.com}                               % optional, remove / comment the line if not wanted
\homepage{stat.wisc.edu/~erker/}                         % optional, remove / comment the line if not wanted
\social[github]{tedwarderker}                              % optional, remove / comment the line if not wanted
\title{title}                               % optional, remove / comment the line if not wanted
\usepackage{lastpage}
%\rfoot{\addressfont\itshape\textcolor{gray}{Page \thepage\ of \pageref{LastPage}}}
\date{\today}
\title{~}
\hypersetup{
 pdfauthor={Tedward Erker},
 pdftitle={~},
 pdfkeywords={},
 pdfsubject={},
 pdfcreator={Emacs 26.1 (Org mode 9.1.7)}, 
 pdflang={English}}
\begin{document}

\maketitle

\section*{Summary of Qualifications and Skills}
\label{sec:orgb1a8fc2}
Biometry M.S. and Forestry Ph.D. (expected spring 2019) with 5 years
of research and statistical analysis experience

\section*{Education}
\label{sec:orgd3c775f}
\cventry{2013--Present}{Ph.D.}{Universifty of Wisconsin--Madison}{}{\textit{3.929}}{Forestry, Department of Forest and Wildlife Ecology\newline{}Committee: Phil Townsend, Jun Zhu, Chris Kucharik, Eric Kruger, Annemarie Schneider.}
\cventry{2013--2018}{M.S.}{University of Wisconsin--Madison}{}{}{\href{https://www.stat.wisc.edu/masters-biometry}{Biometry}, Department of Statistics}
\cventry{2006--2010}{B.A.}{Washington University in St. Louis}{}{\textit{3.83}}{Environmental Studies--Ecology/Biology, Summa Cum Laude}
\section*{Relevant Graduate Coursework and Workshops}
\label{sec:orga82d01d}
\cvitemwithcomment{2017}{Hierarchical Modeling and Analysis of Spatial-Temporal Data}{Workshop}
\cvitemwithcomment{S 2016}{Multilevel Models (STAT 679)}{A}
\cvitemwithcomment{}{Software Carpentry}{Workshop}
\cvitemwithcomment{F 2015}{Statistical Methods for Spatial Data (STAT 575)}{AB}
\cvitemwithcomment{Su 2015}{Statistical Consulting (STAT 699)}{A}
\cvitemwithcomment{S 2015}{Teaching Statistics (STAT 692)}{A}
\cvitemwithcomment{}{Intro Mathematical Statistics II (STAT 312)}{AB}
\cvitemwithcomment{}{Remote Sensing Digital Image Processing (ENVIR ST 556)}{A}
\cvitemwithcomment{}{Tools for Reproducible Research (BMI 826)}{A}
\cvitemwithcomment{F 2014}{Intro Mathematical Statistics I (STAT 311)}{A}
\cvitemwithcomment{S 2014}{Statistical Methods for Bioscience II (STAT 572)}{A}
\cvitemwithcomment{}{Advanced Data Analysis with R (STAT 692)}{A}
\cvitemwithcomment{}{Intermediate Data Analysis with R (STAT 692)}{A}
\cvitemwithcomment{}{Teaching Biology: Special Topics (PL PATH 801)}{A}
\cvitemwithcomment{}{Inquiry-Based Biology Teaching (PL PATH 800)}{A}
\cvitemwithcomment{F 2013}{Statistical Methods for Bioscience I (STAT 571)}{A}

\pagebreak
\section*{Research Experiences}
\label{sec:orgb3bd02b}
\cventry{2015--Present}{Research Assistant}{UW-Madison}{}{}{%
\begin{itemize}
\item Map Urban Forests of Wisconsin
\begin{itemize}
\item Tested 3 machine learning algorithms to classify terabytes of imagery
\item Processed imagery in parallel at UW's Center for High Throughput Computing
\item Geospatial analysis in R and image segmentation in python.
\end{itemize}
\item Carbon Budget of Urban Forest
\begin{itemize}
\item Assessed impact of tree canopy on residential building energy use
and carbon emissions of \textasciitilde{}30,000 Madison homes.
\end{itemize}
\item Canopy Foliar Trait Mapping with \href{https://aviris-ng.jpl.nasa.gov/}{Imaging Spectroscopy}.
\begin{itemize}
\item Applied partial least squares regression models to predict foliar
canopy traits (e.g.  nitrogen content) from \href{https://aviris-ng.jpl.nasa.gov/}{imaging spectroscopy}
data
\item Explored anthropogenic and environmental drivers of trait variation
across Madison, WI.
\end{itemize}
\end{itemize}
}

\cventry{May--Aug 2008}{Research Assistant}{\href{}{Tyson Research Center}}{Eureka, MO}{}{
\begin{itemize}
\item Sampled vegetation, identified over 100 plant species as part of
\href{https://esajournals.onlinelibrary.wiley.com/doi/abs/10.1890/12-1310.1}{study} to explore phylogenic relationships in invasiveness.
\end{itemize}
}

\section*{Statistical Consulting  Experiences}
\label{sec:orgf1667db}
\cventry{2015}{Student Statistical Consultant}{CALS Statistical Consulting Lab}{}{}{%
\begin{itemize}
\item N\(_{\text{2}}\)O flux from maize fields
\begin{itemize}
\item Determined the best time of day to measure N\(_{\text{2}}\)O flux in order to
predict total daily flux.
\end{itemize}
\item Bee foraging behavior
\begin{itemize}
\item Visualized individual bee length of time inside and outside hive
\item Advised on how to test for differences between honey and bumble bee behavior
\end{itemize}
\end{itemize}
}

\cventry{2015-Present}{Informal within Lab Consulting}{Townsend Lab}{}{}{%
\begin{itemize}
\item Designed sampling for study of leaf traits across species and seasons
\item Compared algorithms for predicting leaf foliar traits from
reflectance spectroscopy
\end{itemize}
}

\section*{Teaching Experiences}
\label{sec:orgf98a028}
\cventry{2013--2015}{Teaching Assistant}{UW-Madison}{}{}{
\begin{itemize}
\item Stat 572: Statistical Methods for Bioscience II, Spring 2015
\begin{itemize}
\item Led 2 weekly discussion groups, graded homework and exams for
this graduate-level course largely covering multiple linear and
logistic regression
\item Rated 4.53 / 5.00 in student evaluations (n = 46)
\item Prof. Murray Clayton.
\end{itemize}
\item FWE 550/551: Forest Ecology, Fall 2013 and Fall 2014
\begin{itemize}
\item Redesigned, created and independently implemented lab lessons in
field and computer lab for \textasciitilde{}70 students.
\item Prof. Tom Gower (2013) and Prof. Phil Townsend (2014).
\end{itemize}
\item FWE 110: Living With Wildlife, Spring 2014
\begin{itemize}
\item Graded journals and exams, assisted students during office hours.
\item Prof. Stan Temple.
\end{itemize}
\end{itemize}
}

\cventry{2010--2012}{Chemistry and Biology Teacher}{Confluence Prep Academy}{St. Louis}{}{
\begin{itemize}
\item Educated over 120 students in six classes daily.
\item As first year teacher, developed chemistry curriculum for new charter school integrating College Readiness Standards with Missouri Science Standards.
\item Cross-country coach
\end{itemize}
}

\cventry{2010--2012}{Corps Member}{Teach For America}{Chicago \& St. Louis}{}{
\begin{itemize}
\item Selected from over 46,000 applicants nationwide
\item Commited two years to teach in under-resourced public schools
\end{itemize}
}

\cventry{Apr--Aug 2009}{Farm Education Intern and Farmer}{\href{https://farmandwilderness.org/}{Farm And Wilderness}}{Plymouth, VT}{}{
\begin{itemize}
\item Organized and guided trips of 16-40 students at farm and wilderness education center.
\item Managed 3⁄4 acre garden and cared for sheep, goats, chickens, pigs, and cows as part of farm team.
\end{itemize}
}

\cventry{Jan--May 2008}{Undergraduate Teaching Assistant}{Washington University in St. Louis}{}{}{
\begin{itemize}
\item Brave New Crops, Environmental Studies 3322
\item Prof. Glenn Davis Stone
\end{itemize}
}
\section*{Publications}
\label{sec:org8a22c57}
\cvitem{}{Erker T., Townsend P.A., Wang L., Lorentz L., and Stoltman A. \textit{(in review)}. A statewide urban tree canopy mapping method. \textit{Remote Sensing of Environment}}
\cvitem{}{Erker T., Townsend P.A., \textit{(in prep).} For much of the US, urban shade trees in residential areas may be an atmospheric carbon source}
\cvitem{}{Erker T., Townsend P.A., \textit{(in prep).} Environmental drivers of urban tree canopy foliar traits derived from imaging spectroscopy}

\section*{Presentations and Posters}
\label{sec:org4999290}
\cvitem{Dec 2018}{For much of the US, urban shade trees in residential areas may be an atmospheric carbon source (abstract submitted). American Geophysical Union Fall Meeting, Washington, D.C.}
\cvitem{Apr 2018}{Functional and Species Diversity of Trees in Urban Streets. NASA Biodiversity and Ecological Forecasting Team Meeting, Washington, D.C.}
\cvitem{Nov 2016}{Mapping Urban Tree Canopy of Wisconsin. Society of American Foresters National Convention, Madison, WI}
\cvitem{May 2016}{How Does the Urban Forest Affect the Urban Heat Island and Building Energy Use? NASA Biodiversity and Ecological Forecasting Team Meeting, Silver Springs, MD.}

\section*{Awards, Grants, and Fellowships}
\label{sec:org4632d22}
\cvitemwithcomment{
Jan 2018
}{
\href{http://mc-stan.org/events/}{Stan Conference Scholarship}
}{}

\cvitemwithcomment{
2015-2018
}{
\href{https://nspires.nasaprs.com/external/viewrepositorydocument/cmdocumentid=459947/solicitationId=\%7BB6CDCEA6-8EDD-A48A-FAF8-E588F66661C3\%7D/viewSolicitationDocument=1/NESSF15\%20selections.pdf}{NASA Earth and Space Science Fellowship}
}{}


\cvitemwithcomment{
Sep 2016
}{
Wisconsin DNR Research Grant: Mapping Wisconsin's Urban Tree Canopy
}{}

\cvitemwithcomment{
Oct 2016
}{
\href{https://kb.wisc.edu/russell/page.php?id=65402}{George Kress Award for Outstanding Contribution of a Graduate Student}
}{}

\cvitemwithcomment{
May 2016
}{
Travel Award, UW-Madison Department of Forest and Wildlife Ecology
}{}

\cvitemwithcomment{
Mar 2016
}{
\href{http://news.wisc.edu/cool-science-images-2016/\#\&gid=1\&pid=10}{Cool Science Image contest winner, "Madison Lakes"}
}{}

\cvitemwithcomment{
May  2010
}{
\href{http://enst.wustl.edu/program/awards}{Outstanding Overall Achievement in Environmental Studies}
}{}

\cvitemwithcomment{
Jun 2008
}{
\href{https://tyson.wustl.edu/2008}{Tyson Research Center Summer Undergraduate Research Fellowship}
}{}

\pagebreak

\section*{Service}
\label{sec:org072fbb6}
\subsection*{Department and University}
\label{sec:org9afa5b2}
\cvitemwithcomment{2015--2018}{Graduate Student Representative}{Department of Forest and Wildlife Ecology}
\cvitemwithcomment{2017}{\href{https://software-carpentry.org/}{Software Carpentry Volunteer}}{UW-Madison}
\subsection*{Community}
\label{sec:org9563ab1}
\cvitemwithcomment{2014, 2015}{Guest Lab Instructor, Sustainability by the Numbers}{Shabazz High School}
\cvitemwithcomment{2017}{Guest Lab Instructor, AP Environmental Studies}{East High School}
\subsection*{Undergraduate Mentoring}
\label{sec:org0256657}
\newline{}\cvitemwithcomment{
2017
}{
Cheyenne Brandt
}{
Effect of Leaf Area and Tree Canopy on the Urban Heat Island of Madison, WI.
}

\cvitemwithcomment{
2015
}{
Bobby Shepherd
}{
Investigating the influence of the urban heat island on autumn
phenology of \emph{Acer platanoides} with smartphone hemispherical photos.
}

\section*{Languages and Software}
\label{sec:org22bed7f}
\cvitem{Spoken:}{English, Spanish}
\cvitem{Programming:}{R, Python, Stan, SAS}
\cvitem{Other Software:}{Emacs, QGIS, Microsoft Office, git}

\section*{Professional Affiliations}
\label{sec:org7c26b84}
\cvitem{2016--Present}{Society of American Foresters}
\cvitem{2018--Present}{American Geophysical Union}


\clearpage
\recipient{}{1450 Linden Dr\\Agriculture Hall\\Madison, WI 53706}
\date{\today}
\opening{Dear Search Committee Members, }
\closing{Yours Sincerely,}
\enclosure[Attached]{resume}          % use an optional argument to use a string other than "Enclosure", or redefine \enclname
\makelettertitle

I am applying to the applied statistician position in the CALS
Biometry Statistical Consulting Facility because I'm passionate about
people and statistics.  I know first-hand the dedication researchers
have for their work, and I'm excited by the opportunity to help them
apply statistical methods to gain insight into new problems.  From my
time as a student statistical consultant, I learned the process is
driven first by human considerations, meeting people at their level of
knowledge and recognizing their feelings towards statistics.  From
Nick Keuler, my consulting supervisor, I learned the importance
of listening. When given space to talk, researchers not only clarify
their own thinking, but also identify objectives and needs for the
consultant to address.  My years as a high school teacher and a TA,
especially for Stat 572, taught me that building relationships based
on trust and respect is the foundation of success, learning, and the
proper application of statistics.

My statistical training at UW-Madison and my dedication to constant
improvement would serve me well as a consultant.  Coursework taught me
how to think probabilistically and how to apply a variety of methods
such as spatial statistics and multilevel models.  My research in
forestry has extended the breadth and depth of methods I use.  For
example, I have compared several machine learning algorithms for
classification and analyzed data with more predictors than
observations.  My biometry project uses Stan to fit nonlinear Bayesian
multilevel models of urban tree growth.  I know how to meet the varied
needs of researchers including modeling for inference, prediction and
experimental design, and how to provide advice on statistical
limitations and assumptions.  When clients need or request methods
unfamiliar to me, I look forward to diving into a new challenge and
learning.

I have extensive experiences working with large datasets, multiple
programming languages, and reproducible workflows. For my
dissertation, I classified 4 terabytes imagery at the UW Center for
High Throughput Computing.  I have used many R packages for big data
and also use smart data management techniques to ease the
computational burden.  I work routinely with R and am familiar with
several other programming languages including Stan and SAS.  I have
not used SAS much outside of coursework, but I would enjoy the
opportunity to improve my SAS skills to meet the needs of clients.
All my work is fully reproducible and version controlled, and I love
sharing literate programming with others.  Outreach seminars on best
statistical and data visualization practices would leverage my
excellent teaching and presenting skills.

I am very grateful to be considered for this opportunity and am
confident I have the skills and mindset to contribute to the
success of CALS Biometry.

\makeletterclosing
\pagebreak
\section*{References}
\label{sec:orge2821a0}
\begin{itemize}
\item Jun Zhu
\begin{itemize}
\item Biometry MS Advisor
\item Professor, Department of Statistics and Department of Entomology
\item jzhu@stat.wisc.edu
\item 608-262-1287 (statistics office), 608-890-3916 (entomology office)
\end{itemize}
\item Phil Townsend
\begin{itemize}
\item Forestry PhD Advisor
\item Professor, Department of Forest and Wildlife Ecology
\item ptownsend@wisc.edu
\item 608-263–9107
\end{itemize}
\item Bret Larget
\begin{itemize}
\item Statistics Instructor
\item Professor, Department of Statistics and Department of Botany
\item bret.larget@wisc.edu
\item 608-262-7979 (statistics office), 608-265-6799 (botany office)
\end{itemize}
\end{itemize}
\end{document}